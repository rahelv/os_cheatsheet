\section*{Synchronization}
Cooperating P's: execute concurrently, may be interrupted, share data. Maintain data consistency through Sequencing and Coordination.
% access: LAS or shared mem/message passing
\subsection*{Race Conditions}
Execution outcome dependent on order of concurrent access to shared data (ex: counter in prod-cons-prob, PID)
\begin{description}
        \item[solutions:] disabling interrupts (single core vs. multiprocessors (time-consuming), affects system clock)
        \item[preemptive kernel:] P preempted in kernel mode, most common. more responsive, suitable for real-time programming
        \item [non-preemptive kernel:] uncommon, P blocks CPU
\end{description}

\subsection*{Critical Section Problem}
\begin{description}
    \item[critical section]is where a P accesses shared data (Structure: entry, critical, exit, remainder)
    \item[solution:]mutual exclusion (no 2 P in CS at the same time), progress (selection cannot be postponed indefinitely), bounded waiting (limit on waiting $\rightarrow$ Starvation)
    \item[Peterson's Solution]2 Ps, int turn, bool flag[2] shared, acquire \& lock, not guaranteed to work on modern computer architectures (requires atomic load/store, no instruction reordering)
    \item[Mutex Lock] acquire() and release() lock (atomic), bool available (binary), require busy waiting (spinlock, no context switch required)
    \item[Semaphores] integer variable, accessed through wait() and signal(). busy wait \\
        - \textbf{counting} (init to nr of resources available, decr. in wait) vs. \textbf{binary} semaphore (like mutex lock). \\
        -  \textbf{implementation with suspension and waiting queues} instead of busy waiting S suspends itself, each semaphore has associated waiting queue. signal() removes one P from list and awakens it.   %TODO: Code Example S. 27
    \item[Monitors] high level form of P sync. (ADT, internal vars only acessible within procedures) \\
        - \textbf{condition variables} wait (suspends P) and signal (tells P to resume/condition could have changed) on condition var. \\
        - \textbf{mutex locks}: acquire and release are procedures in monitor
    \item[Priority Inversion] low-prio P holds lock needed by high-prio P. solution: \textbf{priority-inheritance protocol}: inherit higher priority until finished with resources
\end{description}

\subsection*{Bounded Buffer in Producer-Consumer-Problem}
Buffer with n slots, each can hold one item
\begin{description}
    \item[Semaphore Solution]3 S, init: mutex=1, full=0 (nr of fulls slots), empty=n.
    \item[- producer]wait(empty);wait(mutex); //produce signal(mutex);signal(full);
    \item[- consumer]wait(full);wait(mutex) //consume signal(mutex);signal(empty);
\end{description}

\subsection*{Dining Philosophers}
Allocate several resources among several Ps in a DL- and starvation-free manner. ex: 5 ph, 5 forks, 3 states (thinking, hungry, eating). Init: 1 data set (bowl of rise), chopstick[5], initialized to 1 (available).
\begin{description}
    \item[Simplest Solution] remove one ph
    \item[Asymmetric Solution] odd/even pick up chopsticks asymmetrically
    \item[Monitor Solution]cond.var self[5], fun test(i) if hungry \& neighbours not eating $\rightarrow$ self[i].signal() \\
        - \textit{pickup(i)} set i to hungry, call test(i). if not eating afterwards, self[i].wait() \\
        - \textit{putdown(i)} set i to thinking, test left and right neighbour \\
        (starvation (requirement 3) still possible, can be solved by introducting time restriction)
\end{description}

\subsection*{Synchronization within the Kernel (Linux)} %
\begin{description}
    \item[atomic\_t]atomic integers, operations performed without interruptions
    \item[spinlocks]for SMPS, nonrecursive, short locking %TODO: S. 19
    \item[mutex locks, semaphores]
\end{description}

\subsection*{POSIX Synchronization}
    \textbf{pthread.h} API is OS-independent (unix, macOS). provides mutex locks, named (accessible by multiple P's) and unnamed (need to be placed in shared mem) semaphores (include semaphore.h), condition variables (associated with a mutex lock).

\subsection*{Deadlocks}
\begin{description}
    \item[Deadlock]2 or more P's are waiting indefinitely for an event that can be caused only by one of the waiting Ps. 4 conditions: \textbf{Mut. Ex.}, \textbf{Hold and wait} P that holds min. 1 res waits for another res. \textbf{No preemption} res can only be released voluntarily by P holding it. \textbf{Circular Wait} Cycle in res-alloc graph
    \item[Resource-Allocation Graph] Cycle necessary \& sufficient condition (possibility if several instances) %TODO: add picture
    \item[Handling Deadlocks]\textbf{1}. Ensure that sys never enters DL: DL prevention/avoidance ( allow res-alloc only if no DL could happen) \textbf{2}. Allow DL, detect and recover from it. \textbf{3}. Ignore DLs (approach of most OS, user is responsible)
    \item[Deadlock Prevention]: Eliminate at least 1 cond (only D4 practical to eliminate) \\
    \textbf{D1:} Use shareable res (e.g. read-only files) \textbf{D2:} Only req. res if P doesn’t hold other res (problem: low res util/ starvation) \textbf{D3:} (if res not available: first free all, restart if all needed res can be acquired at once) \textbf{D4:} Impose total ordering on all res-types. Requests allowed only in increasing order of enumeration.
    \item[Livelock] P or T continuosly attempts an action that fails. (failure to succeed)
    %TODO: insert example of deadlock with mutex locks
\end{description}

% starvation
