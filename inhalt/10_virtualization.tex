\section*{Virtualization}
Abstract the HW of a single computer (CPU, RAM, Storage) into different execution environments. \\
\begin{description}[labelsep=0pt]
    \item[] \textbf{Host} underlying HW system. \textbf{Hypervisor/VM Manager} provides interface identical to host. \textbf{Guest}
    \item[History] \hspace{2pt} \textbf{Multicomputer Model} Each computer provides different service (+ reliable, isolation (sandboxing), - Maintenance, Scalability) \textbf{Virtualization} overcomes limitations. run multiple OS on one machine, Prototyping, support checkpointing and VM Migration. %TODO: checkpointing, vm migration ?
\end{description}
\begin{description}
    \item[Implementation of Hypervisors]:
        \begin{itemize}[label={}, labelsep=0pt]
        \item \textbf{Type 0}: HW-based solutions, support VM creation and management through \textit{firmware} (VMM itself is encoded in firmware and loaded at boot-time) %limited features bc HW
        \item \textbf{Type 1}: OS-like software or OS's, VMM runs in kernel mode (ex: Windows HyperV, ..), common in data centers, live migrations (balance performance, efficiency), snapshots, cloning.
        \item \textbf{Type 2}: applications that run on standard OS (VMM is a process), limited hw feature exploitation. (ex: VirtualBox) %TODO: evt. add comparison picture type 1 & type 2
        \item \textbf{Emulators}: allow apps written in one system architecture to run on different system arch.
        \item \textbf{Programming Environment Virtualization}: no virtualization of real HW, but creation of optimized virtual system (ex: .NET, JVM)
        \item \textbf{Paravirtualization}: avoids causing traps by modifying guest OS source code
    \end{itemize}
\end{description}
\subsection*{Building Blocks}
Exact Duplicate of HW difficult to provide (esp. with dual-mode operation (kernel/user mode))
\begin{description}
    \item[VCPU]Virtual CPU to represent the state of CPU per guest (as guest believes it to be).  %does not execute code, information from VCPU used during context-switches
    \item[Trap-and-emulate] virtual user mode and virtual kernel mode (guest runs in physical user mode). privileged instruction (ex. sys calls) $\rightarrow$ trap to VMM $\rightarrow$ VMM emulates action $\rightarrow$ return control to guest (kernel mode slower $\rightarrow$ hw support)
    \item[Binary Translation] For CPUs that do not cleanly differentiate privileged instructions. VCPU in user mode $\rightarrow$  run instructions natively. VCPU in kernel mode $\rightarrow$ inspect next few instructions, \textit{Special Instructions} are translated and executed.
    \item[Hardware Support]enables more stable, faster and feature rich virtualization. more CPU modes, hw support for Nested Page Tables, DMA, interrupts (ex: Intel VT-x, AMD: AMD-V)
\end{description}
\subsection*{Virtualization and OS Components}
How do VMMs provide core OS-functions ?
\begin{description}
    \item[CPU Scheduling]VMM acts like multiprocessor system, schedules phys. CPU to VCPUs (using algos). %2 scenarios: enough CPUs, Overcommited, can cause clock drift
    \item[Overcommitment]Guests are configured to use more CPUs/Memory than physically available.
    \item[Memory Management] More Users of Memory $\rightarrow$ more pressure. Overcommitment common. VMM maintains \textbf{Nested Page Table} that translates guest page to real page table. (optimize without user knowing, own page-replacement-algos)
    \item[Storage Management] need to provide boot disk \& general data access. (support many guests, so partitioning not sufficient)
        \begin{itemize}[label={}, labelsep=0pt]
            \item \textbf{Type 0}: store guest root disks \&config as disk image in FS provided by VMM
            \item \textbf{Type 1}: store as files in FS provided by host OS %TODO: evt. mehr infos (S. 27)
        \end{itemize}
    \item[Live Migration]copy running guest to another system. (interesting: MAC must be movable (network). Limitations: Disks cannot be moved. (solve: make remote) \\
    src connects to T(target) $\rightarrow$ T creates guest $\rightarrow$ send readonly pages $\rightarrow$ send read-write pages (clean) $\rightarrow$ send dirty pages (modified, repeatedly) $\rightarrow$ send VCPU's final state and start T $\rightarrow$ terminate source
\end{description}
\subsection*{Containerization}
\begin{description}
    \item[Application Containment]run applications in isolated environment. create virtual layer between OS and applications. Each zone has own applications, Network Stack, Addresses, ... CPU and RAM divided between zones.
    \item[Containers]Standardized Packaging for Software and its Dependencies. Multiple instances, Portable, Isolated (Security), Less Ressource Usage, Quick Startup, Fast(er) Live Migration, Consistent and Reliable Running Env.
    \item[Docker]: %Controlled Namespace for Host Process Environments
        \begin{itemize}[label={}, labelsep=0pt]
            \item \textbf{Image}: represents full applications (store app)
            \item \textbf{Container}: application service location and execution (run app)
            \item \textbf{Engine}ju: Creates, ships and runs Docker containers
            \item \textbf{Registry Service (Docker Hub)}: CLoud or server-based storage \& distribution service for images
            \item \textbf{Dockerfile}: Commands that build Image layer-by-layer (ex: alpine $\rightarrow$ python $\rightarrow$ ..)
        \end{itemize}
        %TODO: add comparison image VMs vs. Containers
\end{description}
