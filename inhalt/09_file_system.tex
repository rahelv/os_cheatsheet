\section*{File Systems}
mechanism for storing and accessing data and programs. \textbf{Files} contain data, \textbf{Directory structure} organizes and stores information about files.

\subsection*{Files}
\textbf{OS View:}  named collection of related info, logical storage unit, \textbf{User View:}  smallest unit to store info in sec. storage.
\begin{description}
    \item[Attributes](symbolic) \textbf{Name}, \textbf{Identifier},  \textbf{Type} (only if OS supports diff. types), \textbf{Location} (pointer to device + location in device), \textbf{Size} (in bytes/words/blocks), \textbf{Timestamps}, \textbf{Protection} (r, w, m)
    \item[Operations]: \textbf{Create}: 1. find space 2. make entry in dir. \textbf{Open}: 1. evaluate file name 2. check access permissions (all ops ex. create\& delete call open) \textbf{Write(filehandle, data to write) / Read(filehandle, pointer in mem to store data) }: keeps position pointer where next read/write must happen. 1 position pointer per process. \\
    \textbf{Reposition / Seek}: changes position pointer, \textbf{Delete(dir, file name)}: releases allocated space, \textbf{Truncate}: erase contents of file
    \item[Structures]: File types can indicate internal structure of files. (ex: ELF (executable and linkable file in Linux)), makes OS large and cumbersome
    \item[Types - File extension]: extensions are used to indicate file types (ex: exe, c, xml, ... )
\end{description}

\textbf{Access Methods} Ways to retrieve/deal with information
\begin{itemize}
    \item \textbf{Sequential}: information processed in order (read\_next() or write\_next()), developed for tape
    \item \textbf{Direct}: allow random access to any file block (file = sequence of records/blocks), developed for disk storage, read(n) or write(n) where n = block number
    \item \textbf{Indexing}: index contains pointers to blocks, search for a record in the file in index. (index can be kept in mem for faster access)
\end{itemize}

\subsection*{Directory}
collection of nodes containing information about all files. (symbol tables that translate file names into file control blocks)
\begin{description}
    \item[Operations]Search, Create, Delete, List, Rename, Traverse
    \item[Structure] \textbf{Single-Level} one dir for all users (grouping and naming problems), \textbf{Two-Level} separate dir for each user (UFD, efficient search, no grouping), \textbf{Tree-based} efficient search, grouping, abs/rel paths \textbf{Acyclic-graph} easy \& good traversal algos, shared subdirs and files, need to guarantee no cycles, complicate searching and deletion \textbf{General Graph} allows cycles, requires garbage collection to recover unused disk space %TODO: explain Multiple user file dir, no cycles bf of possibility of infinite loop
    \item[Memory Mapped Files] Map disk block (phys. mem) to page in virtual mem to directly access file on disk
\end{description}

\subsection*{Implementation}
\begin{description}
    \item[File System Structure]2 Problems: Define User View, Create Algorithms and Data Structures to map logical file system to physical secondary storage devices.\\
        - \textbf{Disks}: rewritten in place, I/O transfers in units of blocks, direct access to any block of info. \\
        - \textbf{NVM}: Non-volatile Memory \\
    \item[Layered File System]\textbf{Application Programs} $\rightarrow$ \textbf{Logical File System} manages metadata and directory structure via \textit{FCB} $\rightarrow$ \textbf{File Organization Module} tracks files and their logical blocks, includes free-space manager $\rightarrow$ \textbf{Basic File System} issues generic (read, write blocks) commands to device driver $\rightarrow$ \textbf{I/O Control} device drivers and interrupt handlers, transfer information between mem and dev
    \item[On-Storage Structures] (Control Blocks) \textbf{Boot CB}: per volume, contains info how to boot OS from that volume. can be empty. \textbf{Volume CB}: per volume, contains volume details (nr of blocks, size of blocks, free-block count and pointers.) \textbf{File CB}: per file, organizes file \textbf{Directory Structure} organize files  %Volume CB: superblock in UFS, NTSF stored in master fil table. FCB: permissions, dates (create, access, write), owners and groups, file size, data blocks TODO: add image
    \item[In-Memory Structures]: \textbf{Mount Table} info about mounted volumes \textbf{Dir-Structure Cache} info of recently accessed dirs \textbf{system-wide open-fil table} contains copy of FCB of each open file \textbf{per-process open-fil table} contains pointers to appropriate entries in sys-wide table \textbf{Buffers} hold FS-blocks %mounting a FS = making it accessible in the linux directory tree
    %TODO: S. 10 (?) Implementation
    \item[Directory Implementation] \textbf{Linear List} easy to program, time-consuming in execution (requires linear search). optimizations: sorted list (but requires sort), binary tree. \textbf{Hash Table} linear list with hash data structure, decreases search time. beware of collisions. fixed size. optimizations: chained-overflow hash table.
    \item[Allocation Methods] \textbf{Contiguous}: file occupies contiguous blocks on device. Simple (needs only block nr \& length), performant. Problems: find space for file, knowing file size, external frag. Requires Compaction (Downtime). %Extents
        \textbf{Linked}: File = linked list of storage blocks. solves problems of cont. frag. dir contains pointer to first\&last blocks. no direct access. (\textit{File Allocation Table FAT}) %TODO: FAT: only keeps track of file allocation but not free space
        \textbf{Indexed}: Each file has index block with pointers to data blocks. Random access, no ext. fragmentation, but overhead for index blocks (too big = overhead, too small = limits file size). \textit{Linked Scheme} link several index blocks, \textit{Multilevel Index} multiple levels of index blocks , \textit{Combined scheme}
    \item[Free-Space Management] File System maintains free-space list. Implementation: \textbf{Bit Vector/Map}: Each block represented by 1 bit, 0 = free, search for first 0 to find free space. \textbf{Linked List}: Link all free blocks together. Traversing time-consuming but seldom needed (always use first block). (\textbf{Grouping}: first free block contains adresses of another n free blocks)..
\end{description}

\subsection*{Internals}
\begin{description}
    \item[Storage] Devices $\rightarrow$ Partitions $\rightarrow$ Volumes $\rightarrow$ File Systems
    \item[Mounting] FS must be mounted before usage. \textbf{Mount point}: Location in dir structure where FS will be attached. Root partition gets mounted by the \textbf{boot loader} (set of blocks that contain enough code to know how to load the kernel) at boot time.
    \item[Partitions] volume containing a FS. \textbf{Cooked} with FS, \textbf{Raw} w/o FS, raw sequence of bytes, \textbf{Root} contains the OS.
\end{description}
