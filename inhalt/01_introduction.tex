\section*{OS Basics}
\begin{description}
  \item[OS] Intermediary b/w user \& HW, executes user programs. convenience, control and coordinate HW  (efficiency).
  \item[OS Services] user interface; prog execution; I/O operations; file-system manipulation; communication between Ps; error detection; res alloc; accounting – who uses how much of what; protection and security
  \item[Direct Memory access] Allow I/O devices to transfer data directly to/from main mem without involving CPU. (1 Interrupt per Data block) %NOTE: instead of 1 per byte (cpu cache)
  \item[Interrupt] request for processor to interrupt current executing code. Processor suspends activities, saves its state and starts executing/gives control to \textbf{Interrupt Service Routine}. \textbf{Interrupt Vector}: contains addresses of all (interrupt) service routines. OS is interrupt driven.
  \item[Trap, Exception] Software-generated interrupt. Caused by software error, system call, other process problems.
  \item[OS data structures] OS needs Lists, stacks queues, trees, maps
\end{description}

\subsection*{Multiprocessor (MP) Systems}
\begin{description}
\item[Advantages]Increased Throughput, Economy of scale, Reliability
  \item[Generic Approach] Each processor performs all (types of) tasks. OS shared among CPUs, each CPU has local private copy of OS data structures.
  \item[Asymmetric MP] Each processor is assigned a special task. Master CPU runs OS, other Slave CPUs run user processes.
  \item[Symmetric MP] Each processor performs all (types of) tasks. OS shared among CPUs. (Lock on OS)
  \item[Non-Uniform Memory Access (NUMA)] Interconnected CPUs each with private mem. They logically share one physical mem space.
  \item[Clustered Systems] Like MP, but multiple computers working together. Linked via some kind of network %NOTE: (e.g LAN).
\end{description}

\subsection*{OS Operations}
\begin{description}
    \item[Bootstrap Program] Initializes system, loads OS kernel and starts execution at power-up. (Stored in \textit{Firmware}). \hfill Degree of Multiprog = Nr. of Ps in mem.
  \item[Batch System, Multiprog.] schedule jobs so CPU always has one job to execute (keep job queue in mem).
  \item[Timesharing, Multitasking]  fast switching between jobs. interactivity for user(s). illusion of concurrency
  \item[Dual-Mode]User/Kernel. Distinguish whether system is running user or kernel code with a HW provided \textbf{Mode bit}. privileged instructions only run in kernel mode; sys. call $\rightarrow$ kernel mode. return $\rightarrow$ user mode. %NOTE: Timer (prevent $\infty$)
\end{description}
