\section*{OS Basics}
\begin{description}
  \item[OS] Program b/w user \& HW, executes user programs, gives convinient usage, ensures efficient use of HW. 
  \item[OS Services] user interface; prog execution; I/O operations; file-system manipulation; communication between Ps; error detection; res alloc; accounting – who uses how much of what; protection and security
    \item[System Call Interface] Programming Interface to services provided by OS written in high-level lang (C or C++), below user interface. Each system call associated with a number, sys. call interface maintains a table of those numbers, calls Kernel to execute and returns status and output.
  \item[Direct Memory access] Load data from/to I/O devices (e.g SSD) directly to/from main memory without involving CPU.
  \item[Interrupt] request for the processor to interrupt currently executing code. The processor will suspend current activities, save its state and execute a function called \textbf{Interrupt Service Routine} which function address is accessed by \textbf{Interrupt Vector}. OS is interrupt driven.
  \item[Trap, Exception] Software-generated interrupt. Caused by software error, system call, other process problems.
  \item[OS data structures] OS needs Lists, stacks queues, trees, maps
\end{description}

\subsection*{Multiprocessor (MP) Systems}
\begin{description}
  \item[Generic Approach] Each processor performs all (types of) tasks. OS shared among CPUs, each CPU has local private copy of OS data structures.
  \item[Asymmetric MP] Each processor is assigned a special task. Master CPU runs OS, other Slave CPUs run user processes.
  \item[Symmetric MP] Each processor performs all (types of) tasks. OS shared among CPUs.
  \item[Non-Uniform Memory Access (NUMA)] Interconnected CPUs each with private memory. They logically share one physical memory space.
  \item[Clustered Systems] Like MP, but multiple computers working together. Linked via some kind of network (e.g LAN).
\end{description} 

\subsection*{OS Operations}
\begin{description}
  \item[Bootstrap Program] Initializes system, loads OS kernel and starts execution at power-up.
  \item[Batch System, Multiprogramming] multiple proc. in mem.; CPU changes process when waiting (for I/O)
  \item[Timesharing, Multitasking]  fast switching between proc; interactive; illusion of concurrency
  \item[Dual-Mode] User mode, kernel mode. Goal: distinguish whether system is running user or kernel code with a HW provided \textbf{Mode bit}; some instructions
privileged, only in kernel mode; sys. call changes mode to kernel; return from call resets it to user
\end{description}
