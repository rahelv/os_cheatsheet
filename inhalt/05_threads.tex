\section*{Threads}
% TODO: process creation is heavyweight, thread creation is light-weight (wieso?)
\begin{description}
  \item[Threads] A basic unit of CPU utilization, executed within P.  Shared with P: code, data, files. Private: registers, stack, PC and own copy of data in T-local-storage \textbf{TLS} (i.e. static data) %NOTE: TLS $\neq$ local variables.
  \item[Multithreaded benefits] responsiveness, resource sharing easier than with Ps, cheaper/faster than P creation, lower overhead in context switching, scalability (even single P can take advantage)
  \item[Multiprocess benefits] isolated, better for tasks that require higher degree of separation and security.
\end{description}

\subsection*{Multicore Programming}
\begin{description}
  \item[Concurrency] Multiple tasks making progress, tasks running out-of-order or in a partial order. (Software Parallelism)
  \item[Parallelism] Requires Concurrency and implies system can run more than one task simultaneously on multiple cores/nodes. (Hardware Parallelism)
  %NOTE: challenges of parallelism: parallelization, partitioning, dependencies, balance, testing/debugging
  \item[Data Parallelism] distribute subsets of data across multiple cores, same operation on each core
  \item[Task Parallelism]  Tasks across  multiple cores, each task running unique operation(s).
  \item[Hybrid Parallelism] Combination of task \& data parallelism
  \item[Amdahl's Law] Speedup from N cores compared to 1 core w/ serial portion S: < 1 / (S + (1-S)/N
\end{description}

\subsection*{Multithreading Models}
\begin{description}
    \item[User threads (UT)] Ts in user space, not visible to kernel \& managed w/out kernel support. (ex: pthreads)
  \item[Kernel Threads (KT)] Supported and managed by OS kernel
  \item[Many-to-One] Many UT mapped to one single KT. Not widely used, as one blocking UT blocks all UT and Ts don't run in parallel
  \item[One-to-One] Each UT maps to one KT (two-level-concurrency: Ts in both user and kernel space are concurrent). Number of Ts restricted due to causing overhead in kernel
  \item[Many-to-Many] Many UT multiplexed to many KT.OS creates as many KT as needed. Because of newer CPUs with many Ts, not that relevant anymore (starts to look like 1:1)
  \item[Two-level-Model] M-to-M with one single 1-to-1 exception (needs guaranteed level of service).
\end{description}

\subsection*{Explicit Threading (ex: Pthreads)} %NOTE: POSIX: Portable Operating System Interface Standard. Specification, not Implementation.

\subsection*{Implicit Threading}
\begin{description}
  \item[Thread pool] Creation of a # of Ts that can be assigned to tasks, creation overhead is reduced.
  \item[Implicit Threading] Managing of Ts by frameworks. Programmer only has to identify tasks
\end{description}

\subsection*{Threading Issues}
\begin{description}
\item[Semantics of fork() \& exec()] if T invokes exec(), it replaces whole P with all Ts. fork() sometimes duplicates P with all Ts and sometimes only calling T.
  \item[Signal handling] (Signals are event based messages to a P) Problem: Where should a signal be delivered for a multithreaded P? Either same T is informed (sync), all Ts (async) or special signal T
  \item[Asynchronous cancellation] T terminates immediately
  \item[Deferred cancellation] T periodically checks if it should terminate (recommended)
  \item[Lightweight Process]data structure between KTs and UTs to manage appropriate nr of KTs allocated to app.
  \item[Scheduler activations] provide upcalls - a communication mechanism from the kernel to the app, to inform the app about events.
\end{description}
